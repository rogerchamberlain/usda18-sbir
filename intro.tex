\section{Introduction}
\label{sec:intro}

The Internet of Things (IoT) is ushering in an era where significant numbers
of devices that perform monitoring and control functions (e.g., process control,
manufacturing, etc.) are connected via wired or wireless networks.
Modern agriculture is a leader in experiencing this transformation,
with ubiquitous data collection associated with planting, fertilizing,
and harvesting of crops as well as with feeding, environmental control,
and monitoring of livestock.
BECS Technology, Inc., (BECS) is a small business that manufactures 
monitoring and control equipment for a number of markets
(agriculture, aquatics, refrigeration, etc.) and seeks to maximize the
societal benefits of the data made available by this connectivity.
However, the current data sharing models we employ (like most companies)
limits each individual organization (farmer) to seeing only the data
logs of equipment that it directly owns. 
This data model significantly impedes the potential that could
be realized.

The benefits that would accrue from
aggregating data across farms are substantial.
For instance, this can lead to better
feed conversion and
diminished usage of feed, water, and electricity.
We can use machine learning techniques to improve yields as well as
catch health-related issues earlier.
While these opportunities can and do provide real benefit to farmers,
there are challenges related to security, privacy, and data communication that
must be overcome.
Both Kumar and Patel~\cite{kp14} and
Vasilomanolakis et al.~\cite{vdlgw15} describe these challenges as being
pervasive across all the IoT.

We desire that the farmer/owners of IoT-based data share those data with
the broader community (including their peers) for the benefit of all,
but this will only happen if we can assure owners that their data will
be kept private.
This proposal seeks to explore the feasibility of cross-organizational
data sharing by addressing the challenge of data privacy.
Additionally, we explore incentives for farmers to participate as well
as techniques for communicating aggregated data
to farmers to encourage better decision-making.
These latter two items are of particular importance to small and mid-size
farms, which are extremely unlikely to have a data analyst on staff.

Some high-profile examples~\cite{bz06,bk07} have illustrated the
difficulty inherent in maintaining the privacy of individuals whose
data are shared, even when anonymized.
Differential privacy~\cite{dwork11,dr14} provides a formal framework
for ensuring bounds on data leaks about individuals; however,
a number of questions remain about its use in practical settings.

What we propose to investigate in this SBIR project is how to transition
the theory of differential privacy into useful practice on agricultural data,
how to incentivize farmers to participate,
and how to communicate such aggregated data back to the farmers.  
Differential privacy provides strong guarantees that no incremental harm
comes to an individual (in our case, a farmer) if they choose to share
their private data with a common database maintained by a trusted curator.
Statistical queries to that database have added noise included in
query responses to preserve privacy.
This added noise introduces uncertainty to the data,
making effective communication a primary challenge. 

The privacy theory is robust and quite strong.  What remains is transition into
practice, and there are a number of interesting questions that we intend
to address as part of this investigation.
\begin{itemize}
\item Differential privacy's theory depends upon appropriate tuning of a number
of parameters (e.g., $\epsilon$, $\delta$).  How the privacy budget impacts
utility is not well understood.  We will investigate this relationship
empirically in the context of agricultural data sets.
\item How do we provide appropriate incentives to encourage farmers to
provide their data for the common good?
\item How do we communicate to users both the query responses and the
inherent associated uncertainty due to differential privacy?
\item How do we combine the use of differential privacy with alternative
approaches~\cite{ct13}, such as $k$-anonymity~\cite{samarati01,sweeney02} and
$l$-diversity~\cite{mkgv07}?
\end{itemize}
BECS will serve as the trusted curator of the database (it is already serving
that role, providing segregated access to individual farmers of their own data).
We will investigate how previous privacy-preserving machine learning
experience~\cite{acgmmtz16,ss15} translates into the space of agricultural data,
what incentives are appropriate and effective~\cite{hssv15},
and how visualization can be used by
laypersons~\cite{ottley2012visually} in the context of anonymized data.

This proposal is responsive to topic area 8.3~Animal Production and Protection
as articulated in the FY~2018 Request for Applications from NIFA.
Specifically, success in the proposed research will address two of the research
priorities of the USDA within this topic area:
\begin{itemize}
\item Priority 1. Improve production efficiency
\item Priority 3. Improve animal health and well-being
\end{itemize}
The availability of shared data can be a strong driver to benefit both of
the above priorities.

In addition, while we are submitting to topic area 8.3,
we see this proposal as also responsive to topic area 8.12~Small and
Mid-Size Farms.  While larger commercial organizations often have access
to extensive data sets of their own, it is the smaller farms that stand to
benefit the most from aggregating data across farms.

The commercial opportunities are significant.
Currently, farmers pay a monthly fee for a data service provided by
BECS, called Feed-Link, in which each farm's equipment makes periodic
contact with BECS's servers (in the cloud). All the data collected during
that period by the equipment is uploaded to a persistent database (also
maintained in the cloud). Farmers then have access to data collected from
their equipment in a number of forms, an illustration of the Feed-Link
dashboard is described in Section~\ref{sec:agIoT}.

At the simplest level, success in this research will enable farmers to
not only have access to the data they already own, but it will also provide
access to data from a much larger collection of farms (all the while,
preserving the privacy of the farmers that have voluntarily provided
those data). This can be an important commercial opportunity for BECS
in two ways: (1)~data that are more valuable to the farmers will likely
support a higher price point for the monthly data service, and (2)~increased
data value will also increase the number of farmers likely to subscribe.

The successful completion of this research is important commercially on
a much broader scale as well.  While the initial implementation of the
system will target equipment manufactured by BECS, the technology is
potentially useful in a much broader setting.  One can envision a
stand-alone data marketplace in which the aggregation of mutually
contributed data that are then shared (in a privacy preserving way)
with the participants to be very attractive to anyone who can benefit
materially from the collected data set.
